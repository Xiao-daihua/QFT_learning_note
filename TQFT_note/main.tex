% main.tex - 科研项目学习笔记模板
% !TeX root = main.tex
%%%%%%%%%%%%%%%%%%%%%%%%%%%%%% DOCUMENT 
\documentclass[12pt]{report}

%%%%%%%%%%%%%%%%%%%%%%%%%%%%%% PACKAGES

% 中文支持(XeLaTeX 编译)
\usepackage[UTF8]{ctex}
\usepackage{xeCJKfntef}          % 中文字体装饰(下划线等)

% 页面设置
\usepackage[a4paper, left=20mm, right=20mm, top=15mm, bottom=15mm]{geometry}

\PassOptionsToPackage{dvipsnames,svgnames,x11names}{xcolor}
\usepackage{xcolor}


% 数学环境及符号
\usepackage{amsmath, amssymb, amsfonts, amsthm}
\usepackage{mathtools}           % amsmath增强
\usepackage{physics}             % 物理公式快捷命令
\usepackage{braket}              % Dirac符号
\usepackage{bbold}               % 数学黑体
\usepackage{dsfont}              % 另一种数字体
\usepackage[mathscr]{eucal}     % 花体字母
\usepackage{tensor}              % 张量指标管理
\usepackage{simpler-wick}       % Wick记号
\usepackage{mathrsfs}            % 另一种花体字母

% 颜色与图形相关
\usepackage{graphicx}           % 插图支持
\usepackage{float}              % 浮动体控制
\usepackage{tikz}               % 绘图库
\usetikzlibrary{math}           % tikz数学扩展

% 表格与列表
\usepackage{makecell}           % 表格多行换行
\usepackage{multicol}           % 多栏排版
\usepackage{colortbl}           % 表格颜色
\usepackage{enumitem}           % 列表自定义

% 其他辅助
\usepackage{framed}             % 有边框环境
\usepackage{tcolorbox}          % 灵活盒子环境
\tcbuselibrary{breakable}       % 盒子内容分页
\usepackage{thmtools}           % 定理环境管理
\usepackage{thm-restate}        % 定理重述
\usepackage{showlabels}         % 显示标签,调试用(完成后可注释)
\usepackage[normalem]{ulem}     % 下划线、删除线
\usepackage{hyperref}           % 超链接(最后加载)
\usepackage{cleveref}           % 智能引用(紧跟hyperref)

% 自定义宏包
\usepackage{macros}

%%%%%%%%%%%%%%%%%%%%%%%%%%%%%% 自定义命令
\newcommand{\tml}{Teichmüller space}
\newcommand{\hil}{Hilbert space}

%%%%%%%%%%%%%%%%%%%%%%%%%%%%%% BEGINNING OF THE DOCUMENT

\begin{document}

\title{\boldmath TQFT笔记}
\author{X. D. H.}
% \emailAdd{yuliu21012858@gmail.com} % 根据需要开启

\maketitle

\begin{abstract}
这是我对于TQFT的学习使用的笔记!!!
\end{abstract}

\tableofcontents

\chapter{TQFT数学基础}
这里我们记录

\chapter{科研主体内容}
这里我们讨论Weinberg第三章的散射理论。这里我们开始考虑粒子之间的相互作用。考虑粒子从无穷远到很近的地方发生相互作用然后再到无穷远。

\section{In and Out States}

\subsection{无相互作用多粒子态}
对于没有相互作用的多粒子态,我们可以认为是单粒子态的直积。那么对于一个有质量的多粒子态,我们进行Lorentz变换可以知道是:
\eq{\label{eq:multitrans}
    \begin{aligned}
        U(\Lambda,a)&\Psi_{_{p_{1},\sigma_{1},\eta_{1};p_{2},\sigma_{2},\eta_{2};\cdots}}=\exp\left(-ia_{\mu}(p_{1}^{\mu}+p_{2}^{\mu}+\cdots)\right)\\&\times\sqrt{\frac{(\overline{\Lambda}p_{1})^{0}(\overline{\Lambda}p_{2})^{0}\cdots}{p_{1}^{0}p_{2}^{0}\cdots}}\sum_{\sigma_{1}^{\prime}\sigma_{2}^{\prime}\cdots}D_{\sigma_{1}^{\prime}\sigma_{1}}^{(j_{1})}\left(W(\Lambda,p_{1})\right)D_{\sigma_{2}^{\prime}\sigma_{2}}^{(j_{2})}\left(W(\Lambda,p_{2})\right)\\&\times\Psi_{_{\Lambda p_{1},\sigma_{1}^{\prime},n_{1};\Lambda p_{2},\sigma_{2}^{\prime},n_{2};\cdots}}
    \end{aligned}
}
复习一下其中$ W(\Lambda,p) $ 是Wigner Rotation,定义是:\cref{eq:Wignerrotation}也就是:$ W(\Lambda,p)=L^{-1}(\Lambda p)\Lambda L(p). $ 。其中D矩阵是SO(3)群的表示矩阵定义为:\cref{eq:Dtrans}。对于无质量的粒子是另一个表示就是$ \delta_{\sigma,\sigma'} exp(i \sigma \theta(\Lambda,p))$。同样的我们定义一个合理的normalization是:
\eq{
\begin{aligned}
&\left(\Psi_{p_{1}^{\prime},\sigma_{1}^{\prime},n_{1}^{\prime};\,
           p_{2}^{\prime},\sigma_{2}^{\prime},n_{2}^{\prime};\cdots},
       \Psi_{p_{1},\sigma_{1},n_{1};\,
           p_{2},\sigma_{2},n_{2};\cdots}\right) \\
&= \delta^3(\mathbf{p}_1^{\prime}-\mathbf{p}_1)
   \delta_{\sigma_1^{\prime}\sigma_1}
   \delta_{n_1^{\prime}n_1}
   \delta^3(\mathbf{p}_2^{\prime}-\mathbf{p}_2)
   \delta_{\sigma_2^{\prime}\sigma_2}
   \delta_{n_2^{\prime}n_2}
   \cdots \\
&\quad \pm\ \text{permutations}
\end{aligned}
} 
这里面考虑了所有permutation的情况。但是注意,会有$ \pm $是因为会有bosonic的情况以及fermionic的情况。
\rmk{我们注意,上面的delta函数都是动量空间三维的。因为我们考虑的都是【同一质量的粒子对应的量子态】。我们不考虑不同质量的粒子量子态的内积。结果是delta函数也很合理,毕竟所有Hermite算符不同本征值的本征函数是正交的。但是系数是1是因为我们取了一个特殊normalization。这个内容具体请复习\cref{sec:normalization}}
\imp{多粒子态简洁记号}{
    为了进行简洁的书写多粒子态我们一般用这样的一个记号来表示:
\eq{
    (\Psi_{\alpha^{\prime}},\Psi_\alpha)=\delta(\alpha^{\prime}-\alpha)
}
注意这里$ \delta(\alpha^{\prime}-\alpha) $ 是一个记号并不是delta函数。并且积分我们也可以这么写:
\eq{
    \int d\alpha\cdots\equiv\sum_{n_{1}\sigma_{1}n_{2}\sigma_{2}\cdots}\int d^{3}p_{1}d^{3}p_{2}\cdots.
}
所以上面的内积结果completeness equation可以写作:
\eq{
    \Psi=\int d\alpha\Psi_\alpha(\Psi_\alpha,\Psi).
}
}
\rmk{注意,我们之前讨论的无相互作用的多粒子态,我们使用的是能动量四矢量+自旋or helicity进行标记说明\hlr{我们考虑满足这个变换的量子态一定是能动量的本征态。}}
我们考虑一个特殊的时间平移Lorentz变换也就是$ \tensor{\Lambda}{^\mu_\nu} = \delta^\mu_\nu $以及$ a^\mu = (0,0,0,\tau) $。在这个情况下我们的能量可以根据\cref{eq:multitrans}写成:
\eq{
    H\Psi_\alpha=E_\alpha\Psi_\alpha
}
其中$ E_\alpha = p^0_1+p^0_2+... $ 。


\subsection{In and Out States}

\hlr{现在我们不考虑无相互作用的粒子,而是考虑一个有相互作用的散射过程。}但是这个过程有个特点,就是在我们认为在$ - \infty $以及$ +\infty $的时间,粒子相当于没有相互作用的。  

\imp{In and Out的基本思路}{
    描述散射过程我们需要考虑结果和初始。这两个阶段粒子都应该没有相互作用了,并且时间是$ - \infty $以及$ +\infty $的点。
    
    所以对于散射过程我们可以定义两个量子态$  \Psi_\alpha^+, \Psi_\alpha^- $我们希望这两个量子态表征散射的初始状态和结果状态。所以我们赋予要求:
    \itm{
        \pt{这两个量子态对于「在$ \pm\infty $ 时间的观察者来说,等价于$ \Psi_\alpha $的量子态 」}
    }
}
\rmk{
    我们进行三个解读:
    \itm{
        \pt{首先,我们一直使用的是Heisenberg Picture进行描述量子态。\hlr{我们不认为量子态是定义在一个等时面的。而是描述整个时空全部的。量子态是与时间空间无关的向量。}但是我们知道不同观察者观察到的同一个量子态,但是观测到的是不同Hilbert空间上的矢量。}
    }

    一个具体的例子,考虑一个观察者B,他在A的未来时刻$ t_B = t_A - \tau $【注意这里的符号】。当一个事件在A的时间$ t_A = \tau $ 发生的时候,对于B来说这个时间就在$ t_B = 0 $发生。
    
    对于未来的观察者来说,如果A观察者在某一个时刻观察到了一个量子态$ \Psi_\alpha $,那么对于B来说,他在这个事件发生同时观测到的量子态就是$ U(Id,-\tau) \Psi_\alpha = e^{-iH\tau} \Psi$ 。

    这也就是时间平移算符的正负号的来源,请注意。
\itm{
    \pt{第二个值得注意的是,我们既然如此,并不存在绝对的量子态,而是存在不同时间的量子态的相对关系,除非我们定义一个时间点。}
}
但是,这里我们知道不论定义哪个时间点。我们对于负无穷时间的观察者,如果时间发生在自己的时间0点(当然选哪个有限的时间都是一样的。)那么一般的观察者会知道这个事件发生在$ -\infty $的时间点。对于正无穷时间的观察者来说,他会知道这个事件发生在$ +\infty $的时间点。所以$ t_{\mp\infty} = t_A -(\mp \infty) $。这个才是正确的洛伦兹变换,以及怎么理解这件事。    

\itm{
    \pt{还有值得注意的是:什么叫"在$ \infty $时间等价于无相互作用粒子态"。因为我们考虑的量子态都是能动量本征态,那么由于不确定性原理,这样的量子态必然是弥散在时间和空间上面的。}
}

从数学的角度,如果我们直接对于$ \Psi_\alpha $量子态进行时间演化。我们会得到的$  U(Id,-\tau) = e^{-iH\tau}= e^{-E_\alpha \tau} $就是一个相位。是没有意义的,因为相差相位的量子态是同一个量子态。

所以我们需要额外技术解决这个问题。我们将考虑“\hlr{在波包的意义下},量子态在$ \infty $时间等价于无相互作用粒子”。具体写出来就是,我们定义一个波包的superposition的系数是$ g(\alpha) $并要求这个相位满足:【连续的叠加一些在有限范围$ \Delta E $的量子态 】
在这个的基础上我们可以定义设么叫:【在$ \infty $时间等价于无相互作用粒子态 】公式表达是:
\eq{
    \exp(-iH\tau)\int d\alpha\mathrm{~g}(\alpha)\Psi_\alpha^\pm=\int d\alpha\mathrm{~e}^{-iE_\alpha\tau}g(\alpha)\Psi_\alpha^\pm
}
分别对于$ \tau \gg 1/\Delta E $ 和$ \tau $远小于$ -1/\Delta E $
\YL{【不知道为啥,但是远小于符号就是打不出来!!】}

}

在上面充分的讨论的基础上我们就可以严格的定义in and out states了。

\imp{In and Out states严格定义}{
    我们研究一个散射过程之中我们使用的Hamiltonian可以写作下面的形式:
    \eq{
        H = H_0 + V
    }
    其中$ H_0 $代表没有相互作用的部分,而$ V $代表粒子之间的相互作用。并且我们要求:
    \itm{
        \pt{$ H $和$ H_0 $ 必须有一模一样的spectrum。也就是两个Hamiltonian的本征值是一样的。 }
    }
    这个要求本质上就是要求$ H_0 $并不是单纯的$ H $把所有的势能项直接删掉,我们需要定义一个新的质量,保证新的质量定义之下$ H_0 $ 的spectrum和之前的一样。为了定义新的质量减去的一些term要放到V里面去。  
    
    \defi{
      无相互作用多粒子态

      无相互作用的多粒子处于无相互作用的Hamiltonian也就是$ H_0 $ 的本征态,$ \Phi_\alpha $满足:
      \eq{
        H_0\Phi_x=E_x\Phi_x,\quad (\Phi_{x^{\prime}},\Phi_x)=\delta(\alpha^{\prime}-\alpha).
      } 
    }

    由于两个Hamiltonian有一样的spectrum所以我们下面可以定义In and Out state:
\defi{
  in and out states定义

  in and out states是满足下面两个条件的量子态:
 \begin{enumerate}
        \item 【对于$ H $来说 】和$ \Phi_\alpha $ 对于$ H_0 $来说有一样本征值的量子态$ \Phi_{\alpha}^{\pm} $也就是说:
        \eq{
            H\Psi_\alpha^\pm=E_\alpha\Psi_\alpha^\pm
        }
        \item 在$ \infty $等价于$ \Phi_\alpha $态:
        \eq{\label{eq:secondconditiontofreepartical}
            \int d\alpha\mathrm{~}e^{-iE_{x}\tau}g(\alpha)\Psi_{\alpha}^{\pm}\to\int d\alpha\mathrm{~}e^{-iE_{x}\tau}g(\alpha)\Phi_{\alpha}
        }
        分别对于$ \tau \to \mp \infty $。注意$ \Psi^+_\alpha $是in state对应的是$ \tau = -\infty $ 。

    \end{enumerate}

}
        对于这个关系我们有一个形式化的算符表达:
        \eq{
            \exp(-iH\tau)\int d\alpha\mathrm{~g}(\alpha)\Psi_\alpha^\pm\to\exp(-iH_0\tau)\int d\alpha\mathrm{~g}(\alpha)\Phi_\alpha
        }
        这个式子我们可以写成:
        \eq{\label{eq:timeinoutstate}
            \Psi_{\alpha}{}^{\pm}=\Omega(\mp\infty)\Phi_{\alpha},
        }
        其中我们定义:
        \eq{
            \Omega(\tau)\equiv\exp(+iH\tau)\exp(-iH_0\tau).
        }
        注意这个仅仅是一个形式化的表达。等式\cref{eq:timeinoutstate}是在「考虑波包」的意义下成立的。
}
下面我们讨论这样的in and out states满足什么样的性质:
\bigskip

\hlr{性质1: in and out states是互相正交归一的量子态}

我们考虑in和in states或者out 和 out states之间的内积,我们会发现根据\cref{eq:secondconditiontofreepartical}存在in and out states的内积和自由多粒子态的内积之间的关系:
\eq{
    \int d\alpha d\beta\exp(-i(E_\alpha-E_\beta)\tau)g(\alpha)g^*(\beta)(\Psi_\beta^\pm,\Psi_\alpha^\pm)=\\
\int d\alpha d\beta\exp(-i(E_\alpha-E_\beta)\tau)g(\alpha)g^*(\beta)(\Phi_\beta,\Phi_\alpha).
}
由于这个关系对于所有任取的波包 $ g(\alpha) $都成立,所以我们立刻知道只有当满足:
\eq{
    (\Psi_\beta^\pm,\Psi_\alpha^\pm)=(\Phi_\beta,\Phi_\alpha)=\delta(\beta-\alpha).
}
\rmk{我们注意,这个是in和in state或者 out和out state之间内积的结果。我们不考虑in和out state之间的内积。}

\hlr{性质2: in and out states的explicit形式}

其实满足定义的两条关系我们可以形式化的写出一个explicit的形式。我们首先把定义之中的$ H = H_0+V $进行展开:
\eq{\label{eq:formalrewriteinout}
    (E_\alpha-H_0)\Psi_\alpha^\pm=V\Psi_\alpha^\pm.
} 
然后我们意识到,由于$ \Phi_\alpha $ 也是本征值为$ H_\alpha $但是是$ H_0 $ 的本征态,所以我们有:$( E_\alpha-H_0 )\Phi_\alpha = 0 $也就是算符$ ( E_\alpha-H_0 ) $ 湮灭了$ \Phi_\alpha $量子态。于是\cref{eq:formalrewriteinout}可以写作:
\eq{
    (E_\alpha-H_0)(\Psi_\alpha^\pm-\Phi_\alpha)=V\Psi_\alpha^\pm.
}
这个时候我们希望把两边乘以算符$ (E_\alpha-H_0) $的逆。但问题是这个算符并不可逆,因为它的零空间并不是0,存在$ \Phi_\alpha $在零空间里。所以我们使用一个小技巧,我们把这个算符稍微perturb一下:   

\chapter{知识补充!}
这里记录一些补充的知识!!!

\chapter{Scratch Book}
\sout{这里会放一些写的很混沌,但懒得扔掉的东西呜呜呜呜!!!这里会放一些写的很混沌,但懒得扔掉的东西呜呜呜呜!!!这里会放一些写的很混沌,但懒得扔掉的东西呜呜呜呜!!!这里会放一些写的很混沌,但懒得扔掉的东西呜呜呜呜!!!这里会放一些写的很混沌,但懒得扔掉的东西呜呜呜呜!!!}

真的吗???

\sout{这里会放一些写的很混沌,但懒得扔掉的东西呜呜呜呜!!!这里会放一些写的很混沌,但懒得扔掉的东西呜呜呜呜!!!这里会放一些写的很混沌,但懒得扔掉的东西呜呜呜呜!!!这里会放一些写的很混沌,但懒得扔掉的东西呜呜呜呜!!!这里会放一些写的很混沌,但懒得扔掉的东西呜呜呜呜!!!}
哈哈哈哈!!!


\cjkhl{lightblue}{的很混沌,但懒得扔掉的东西呜呜呜呜!!!这里会放一些写的很混沌,但懒得扔掉的东西呜呜呜呜!!!}


\end{document}
