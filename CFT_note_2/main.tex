% main.tex - 科研项目学习笔记模板
% !TeX root = main.tex
%%%%%%%%%%%%%%%%%%%%%%%%%%%%%% DOCUMENT 
\documentclass[12pt]{report}

%%%%%%%%%%%%%%%%%%%%%%%%%%%%%% PACKAGES

% 中文支持(XeLaTeX 编译)
\usepackage[UTF8]{ctex}
\usepackage{xeCJKfntef} 

\setCJKmainfont{HanziPen SC}
% \setmainfont{HanziPen SC}


% 页面设置
\usepackage[a4paper, left=20mm, right=20mm, top=15mm, bottom=15mm]{geometry}

\PassOptionsToPackage{dvipsnames,svgnames,x11names}{xcolor}
\usepackage{xcolor}


% 数学环境及符号
\usepackage{amsmath, amssymb, amsfonts, amsthm,amsopn}
\usepackage{tensor}              % 张量指标管理
\usepackage{mathtools}           % amsmath增强
\usepackage{physics}             % 物理公式快捷命令
             % Dirac符号
\usepackage{bbold}               % 数学黑体
\usepackage{dsfont}              % 另一种数字体
\usepackage[mathscr]{eucal}     % 花体字母
\usepackage{tensor}              % 张量指标管理
\usepackage{simpler-wick}       % Wick记号
\usepackage{mathrsfs}            % 另一种花体字母

% 颜色与图形相关
\usepackage{graphicx}           % 插图支持
\usepackage{float}              % 浮动体控制
\usepackage{tikz}               % 绘图库
\usetikzlibrary{math}           % tikz数学扩展
\usepackage{geometry}
% 表格与列表
\usepackage{makecell}           % 表格多行换行
\usepackage{multicol}           % 多栏排版
\usepackage{colortbl}           % 表格颜色
\usepackage{enumitem}           % 列表自定义

% 其他辅助
\usepackage{framed}             % 有边框环境
\usepackage{tcolorbox}          % 灵活盒子环境
\tcbuselibrary{breakable}       % 盒子内容分页
\usepackage{thmtools}           % 定理环境管理
\usepackage{thm-restate}        % 定理重述
\usepackage{showlabels}         % 显示标签,调试用(完成后可注释)
\usepackage[normalem]{ulem}     % 下划线、删除线
\usepackage{hyperref}           % 超链接(最后加载)
\usepackage{cleveref}           % 智能引用(紧跟hyperref)
\usepackage{soul}

% 自定义宏包
\usepackage{macros}

% 一个中文可以高亮的包
\usepackage{cjkhl}
\definecolor{lightblue}{rgb}{.8,.8,1}

%%%%%%%%%%%%%%%%%%%%%%%%%%%%%% 自定义命令
\newcommand{\tml}{Teichmüller space}
\newcommand{\hil}{Hilbert space}

%%%%%%%%%%%%%%%%%%%%%%%%%%%%%% BEGINNING OF THE DOCUMENT

\begin{document}

\title{\boldmath Algebra in  Braided Tensor Categories  and  Conformal Field Theory学习笔记}
\author{X. D. H.}
% \emailAdd{yuliu21012858@gmail.com} % 根据需要开启

\maketitle

\begin{abstract}
这个笔记是我学习Runkel对于其长篇连续剧的一个简单的概括进行讨论,主要作为「Algebra in  Braided Tensor Categories  and  Conformal Field Theory」的学习笔记!!
\end{abstract}

\tableofcontents

\chapter{Introduction}
我们基本介绍一下我们讨论的对象。

\chapter{QFT in Functors}
\input{QFT_functor.tex}

\chapter{TQFT}
\input{CFTalgebra.tex}

\chapter{Scratch Book}
\sout{这里会放一些写的很混沌,但懒得扔掉的东西呜呜呜呜!!!这里会放一些写的很混沌,但懒得扔掉的东西呜呜呜呜!!!这里会放一些写的很混沌,但懒得扔掉的东西呜呜呜呜!!!这里会放一些写的很混沌,但懒得扔掉的东西呜呜呜呜!!!这里会放一些写的很混沌,但懒得扔掉的东西呜呜呜呜!!!}

真的吗???

\sout{这里会放一些写的很混沌,但懒得扔掉的东西呜呜呜呜!!!这里会放一些写的很混沌,但懒得扔掉的东西呜呜呜呜!!!这里会放一些写的很混沌,但懒得扔掉的东西呜呜呜呜!!!这里会放一些写的很混沌,但懒得扔掉的东西呜呜呜呜!!!这里会放一些写的很混沌,但懒得扔掉的东西呜呜呜呜!!!}
哈哈哈哈!!!


\cjkhl{lightblue}{的很混沌,但懒得扔掉的东西呜呜呜呜!!!这里会放一些写的很混沌,但懒得扔掉的东西呜呜呜呜!!!}


\end{document}
