本章节我们主要follow 大黄书 Conformal Field Theory的基础知识的内容。

\section{Quantization}
这里我们列举一些玻色和费米场的正则量子化的结论,具体细节推导和理解可以之后补充。



我们定义什么是路径积分:
\thm{\hdt{路径积分}

波色场的路径积分为:
    \eq{
        \langle\varphi_f(\mathbf{x},t_f)|\varphi_i(\mathbf{x},t_i)\rangle=\int[d\varphi(\mathbf{x},t)]e^{iS[\varphi]}
    }

    费米场的路径积分为:
    \eq{
        \langle\psi_f(x,t_f)|\psi_i(x,t_i)\rangle=\int[d\bar{\psi}d\psi]e^{iS[\bar{\psi},\psi]}
    }
}



我们讨论一下我们怎么理解路径积分:
\rmk{
    我们认为路径积分其实是对与场的构型的积分。也就是每一个积分的元素是场在全是空的一个分布。

    我们为时空上面每一个点赋予一个场我们称之为\seq{\psi(x,t)},并且根据这个我们可以求出这个构型对应的一个作用量我们定义为:\seq{S[\psi(x,t)]}作用量本质上就是
    这样的一个全时空的场的构型的泛函。

    对于边界条件,也就是我们约束场的每一种积分的构型都需要满足条件:
    \eq{
        \psi(x,t = t_f) = \varphi_f(x) \quad \psi(x,t = t_i) = \varphi_i(x)
    }
    这样的路径积分的结果就是一个数,这个数表示两个态之间的概率。
}

\section{correlation function}
首先我们介绍量子力学意义下面的关联函数是什么,接下来我们推广到量子场论之中:
\subsection{量子力学的关联函数}
我们定义一些定义在一个粒子在不同时间的位置的关联函数是\seq{x(t)},接下来我们可以定义关联函数为:
\defi{我们定义关联函数为:
    \eq{
        \langle x(t_1)x(t_2)\cdots x(t_n)\rangle = \langle0|\mathcal{T}\left(\hat{x}(t_1)\cdots\hat{x}(t_n)\right)|0\rangle 
    }
    特别注意,所有算符必须是time ordered(这个在正则量子化的语境下十分令人困惑),也就是说:
    \eq{
        \mathcal{T}(x(t_1)\cdots x(t_n))=x(t_1)\cdots x(t_n)\quad\mathrm{if}\quad t_1>t_2>\cdots>t_n
    }
}
上方语言是在正则量子化的语境下有些令人困惑,但是推广到路径积分的语境我们可以很自然的发现 time ordered是一个很重要的条件:

接下来我们给出定理,也就是路径积分量子化语境下关联函数的等价表示:
\thm{
    在路径积分量子化的语境之下,关联函数可以写成:
    \eq{
        \begin{array}{rcl}\langle x(t_1)x(t_2)\cdots x(t_n)\rangle&=& \lim_{\varepsilon\to0} \frac{\int [dx]x(t_1)\cdots x(t_n)\exp iS_\varepsilon[x(t)]}{\int [dx]\exp iS_\varepsilon[x(t)]}\end{array}.
    }

    注意:中路径积分的时间必须有一个欧几里得的分量,其中的:
    \eq{
        S_{\epsilon} = \int_{-\infty}^{\infty} dt (1-i\epsilon) \mL(x,\frac{d x}{d t(1-i \epsilon)} , t(1-i\epsilon))
    }
    也就是把积分微元替换成\seq{t(1-i\epsilon)}这个时候拉格朗日量之中的所有含时元素也需要变成\seq{t(1-i\epsilon)}

    

    注意:我们路径积分是对于数的积分而不是对于算符的积分,路径积分之中涉及的\seq{x(t_i)}指的就是粒子在\seq{t_i}时间的位置坐标,由于是量子力学的理论我们考虑的积分空间只有1维时间维。x其实是场!
}

接下来我们说明路径积分出来的关联函数和正则量子化语境下面定义的关联函数是等价的。这里我们归一化正则量子化语境下面的关联函数是:
\eq{
    \langle x(t_1)x(t_2)\cdots x(t_n)\rangle = \frac{\langle0|e^{iHt_1}\hat{x}e^{iH(t_2-t_1)}\hat{x}e^{iH(t_3-t_2)}\cdots\hat{x}|0\rangle}{\langle0|e^{iH(t_n-t_1)}|0\rangle}
}
因为我们的某个时间的算符可以写成:
\eq{
    \hat{x}(t)=e^{iHt}\hat{x}e^{-iHt}
}
\at{
    我们注意,7式子之中我们可以写成这样类似路径积分的形式,是因为我们把算符按照时间的顺序排列了!
}

接下来我们定义研究两个算符在真空态的平均值的比值,我们可以知道:
\eq{
    \frac{\langle0|\mathcal{O}_1|0\rangle}{\langle0|\mathcal{O}_2|0\rangle} = \lim_{T_i,T_f\to\infty}\frac{\langle\psi_f|e^{-iT_fH(1-i\varepsilon)}\mathcal{O}_1e^{-iT_iH(1-i\varepsilon)}|\psi_i\rangle}{\langle\psi_f|e^{-iT_fH(1-i\varepsilon)}\mathcal{O}_2e^{-iT_iH(1-i\varepsilon)}|\psi_i\rangle}
}
因为真空态可以由无穷的欧几里得路径积分得到:
\eq{
    e^{-iT_iH(1-i\varepsilon)}|\psi_i\rangle & =\sum_ne^{-iT_iH(1-i\varepsilon)}|n\rangle\langle n|\psi_i\rangle  \\
&=\sum_ne^{-iT_iE_n(1-i\varepsilon)}|n\rangle\langle n|\psi_i\rangle \\
&\to e^{-iT_iE_0(1-i\varepsilon)}|0\rangle\langle0|\psi_i\rangle\quad\mathrm{if}\quad\varepsilon\to0 , T_i\to\infty 
}

\at{
    这个时候我们就凸显了使用\seq{t(1-i\epsilon)}的好处,也就是可以获得一个欧几里得路径积分来构造我们的真空态。
    从而保证我们的关联函数是对于真空态的平均。

    同时我们注意到,我们使用了\seq{t \to 0}的假设,这个保证了我们拉格朗日量的形式不会发生变化。否则其实我们做路径积分的时候,如果考虑时间演化函数是\seq{exp(-H t(1-i\epsilon))}也就是我们推导过程之中使用的\seq{t \to t(1- i\epsilon)}那么我们其实路径积分指数
    上面的作用量已经不再是对于拉格朗日量的积分,而是拉格朗日量函数形式进行一些些的变化(甚至如果\seq{\epsilon}很大的时候这个函数形式会发生不小变化!!)这个变化产生于我们拉格朗日量部分的求导的变量已经不再是t而变成了\seq{t(1-i\epsilon)}
}


\rmk{
    容易让人困惑的是这个路径积分的边界条件是什么?

    根据我们的推导我们会发现一个有趣的事实,就是不论取什么样的边界条件都不影响路径积分的结果。
    因为上方的式子在推导的时候最后一步可以写成:
    \eq{
        \int dx(t = \infty) dx(t = -\infty) \brakit{\psi_f}{x(t = \infty)} \brakit{x(t = -\infty)}{\psi_i} \int_{x(t = \infty)}^{x(t = \infty)} \mD x(t) \ \mO e^{iS_{\epsilon}}
    }

    那么我们会选用一些我们习惯的边界条件,比如就是取边界条件是一个位置算符的本征值。
    \eq{
    \kit{\psi_i}  =  \kit{x_{0}} \quad \kit{\psi_f} = \kit{x_n}
    }
这里就是涉及一个让人困惑的事情就是当对于真空态取时间演化算符的平均值的时候,也就是求配分函数的时候:
\eq{
    Z[0] = \bra{0} e^{-H t(1-i\epsilon )} \kit{0} = \int_{x_0}^{x_n} \mD x(t) \ e^{iS_{\epsilon}}
}
虽然我们取的是一定时间的算符,但是其实路径积分把这个时间的信息给消磨掉了!我们的路径积分是对于从负无穷到正无穷的时间进行的路径积分!并且边界条件是任取的

这里这么不清楚是因为我们通过了一个加入很小的\seq{\epsilon}的操作混淆了欧几里得路径积分和真实时空的路径积分
我们真实的操作其实是,先通过不确定边界条件的欧几里得路径积分得到了真空态,再进行洛伦兹的路径积分,然后再进行欧几里得路径积分得到真空态。
那么对于配分函数更舒服的表达其实应该看下面的图

}
\pict{2024-07-18-13-27-29.png}{0.7}

\at{
    由于我们知道边界条件和路径积分求关联函数并没有任何关系,那么我们之后求关联函数不再写积分边界条件。因为写边界条件并没有意义!
    
    
    并且我们认为积分永远是对于我们讨论问题的整个流形进行积分!
    }




\subsection{欧几里得路径积分}
我们一般用相对论的路径积分定义一切,但是问题是,这样的路径积分形式很复杂也很难计算。同时写成这样的形式也会导致我们
没有办法从式子之中看出来量子场论和统计力学的联系。

位了更加简便的引入路径积分,我们进行一个变量替换(这个操作的合法性由复分析保证,可以单拿出一章讨论复变函数)我们把所有的(包括求导变量,积分变量)的t
替换为欧几里得时间。注意这样的替换会导致拉格朗日量和作用量的函数形式发生一些些的变化(有的时候拉格朗日量直接变成了哈密顿量!)。这个时候我们给出下面的定理:
\thm{
    我们定义下面的欧几里得(算符,作用量,拉格朗日量)为:
    \eq{
        \hat{X}(-i\tau) = \hat{X}_E(\tau)  \quad iS_E[x(\tau)]=S[x(t\to-i\tau)] \quad L_E(x(\tau))=-L(x(t\to-i\tau)) 
    }
    我们可以给出关联函数等于
    \eq{
        \langle x(\tau_1)x(\tau_2)\cdots x(\tau_n)\rangle = \frac{\int[dx]x(\tau_1)\cdots x(\tau_n)\exp-S_E[x(\tau)]}{\int[dx]\exp-S_E[x(\tau)]}
    }

    这个时候我们就体现出之前边界条件书写不含时间的优越性了,因为我们发现,边界条件与时间没有关系,就是两个空间上面的场的分布。所以
    边界条件在欧几里得路径积分的时候也不会发生变化。
}

值得注意的是,我们改成欧几里得路径积分之后,时间和空间维度并没有任何的区别。当然我们路径积分仍然是对整个时空的所有的构型
进行积分。这里我们只对于t积分是因为我们考虑的是量子力学,时空只有时间维度。

\line
根据上文的讨论接下来我们很容易将欧几里得路径积分得到的关联函数推广到量子场论,因为时间和空间维度是完全等价的:

\eq{
    \langle\phi(x_1)\cdots \phi(x_n)\rangle = \frac{\int \mD \phi\  \phi(x_1)... \phi(x_n) exp(-S_E)}{\int \mD\phi \ exp(-S_E)}
}



\rmk{
    今后我们的讨论使用的计算全部都是对于欧几里得路径积分的计算

    但是容易含糊的是,欧几里得的作用量,拉格朗日量,算符,还有关联函数,我们的函数形式和正常的路径积分完全不一样。
    但是我们知道如果取自变量换元\seq{t = -i \tau}那么得到的数值是一样的!
}



\subsection{怎么计算关联函数}
为了方便的计算关联函数,我们引入Generator funtion的概念,我们定义:
\defi{
    Generator function

    定义生成函数满足下面的式子:
    \eq{
        Z[j]=\int\mD \phi(x) \mathrm{~exp}-\left\{S[\phi(x)]-\int dt\mathrm{~j}(x)\phi(x)\right\}
    }
    对于欧几里得量子场论我们使用的就是\seq{\phi(x)}表示,我们并不区分时间维度和其他维度,可以放心的进行正常欧几里得时空之中的积分。
}
根据这个定义我们可以知道路径积分其实就是\seq{Z[0]}
接下来我们可以发现下方等式成立:
\eq{
    \text{Z[j]}  &=Z[0]\langle\exp\int dtj(t)\phi(x)\rangle \\
    \langle \phi(x_1)\cdots \phi(x_n)\rangle &=Z[0]^{-1}\frac\delta{\delta j(t_1)}\cdots\frac\delta{\delta j(t_n)}Z[j]\bigg|_{j=0}
} 
我们可以使用更直接的求导得到我们的关联函数

\subsection{总结}
这里我需要进行一个总结。我们知道有两种量子化方式,并且这两种量子化方式是等价的,至于怎么样的定价,我认为下方的话可以描述:

1. 路径积分量子化:关联函数是物理量在全平面对于作用量加权的路径积分下的值

2. 正则量子化:关联函数是time ordered(因此为了定义正则量子化我们必须先定义一个正交于空间维度的“时间”)物理量量子化后的算符在真空量子态下面的平均值

上方手段定义的“关联函数”必须是等价的。这个意义上两个量子化方式是等价的,而这种等价也让我们可以在定义一种量子化后,找到相应系统的另一种量子化的定义。

\newpage


\section{Symmetry of Fields}

\subsection{协变经典场}
我觉得协变完全可以用一个式子表达,我们定义一个场是协变的,那么这个场满足下面的性质:

\defi{
    流形上的协变的场A

    我们定义一个群G,以及一个流形上的场。g的元素让流形上的点产生了一些移动和变化\seq{x' = g x}同时也会对场产生一些变化
    \eq{
        A^{\mu'} (x'= g x) = R^\mu_\nu A^\mu(x)
    }
如果满足这个关系,并且R 是G在场空间的表示
}
而在量子场论之中,我们研究的场(标量场,矢量场,旋量场)我们一般认为都是协变的!
也就是说给出一个群使得坐标变换,那么场必然按照这个群的表示变换。后面我们会
给出具体的方式求解群表示。

之后我们要定义一种特殊的场,也就是张量场。张量场我们认为是满足一些特殊变换关系的协变。也就是
张量场在坐标变换之下按照一个特殊的矩阵发生函数形式的变化。

注释:这里的张量场其实就是广义相对论之中讨论的张量场!

\defi{流形上的张量场T


我们认为流形上的张量场在变换之下\seq{x \to x'}协变关系是:

\eq{
    T^{\mu'} |_{x'} = \frac{dx^{\mu'}}{d x^{\mu}} T^{\mu} |_{x}
}
我们会发现我们对于流形上的张量场有两种截然不同的理解,并且这样的两种理解是一模一样的。(对于一般的协变的场,我们只用主动变换理解)

第一种理解:(主动观点)x'是流形微分同胚变换后另一个点的坐标

第二种理解:(被动观点)x'是x换了一个坐标系之后的坐标

}

我们注意到,对于“张量场”这样一种特殊的协变的场,这样的两种理解给出的变换的数学形式是
一模一样的。至于使用哪种理解,我个人认为,被动观点很方便理解计算的细节;主动变换很方便
和协变场统一理解”协变“是什么。 



\subsection{经典协变场守恒流}



\subsection{量子协变场——关联函数协变}


\subsection{量子协变场——Ward identity}
这里